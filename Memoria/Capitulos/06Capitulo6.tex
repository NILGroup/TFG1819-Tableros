%---------------------------------------------------------------------
%
%                          Cap�tulo 6
%
%---------------------------------------------------------------------

\setlength{\parskip}{10pt}
\chapter{Conclusiones y trabajo futuro}

\begin{resumen}
	En este capitulo en la secci�n \ref{cap6:sec:conclusiones} se muestran las conclusiones extra�das del TFG y en la seccion \ref{cap6:sec:trabajofuturo} el trabajo futuro que podr�a implementarse extendiendo la funcionalidad de este proyecto.
\end{resumen}

%-------------------------------------------------------------------
\section{Conclusiones}
%-------------------------------------------------------------------
\label{cap6:sec:conclusiones}

Las personas con discapacidad cognitiva, presentan dificultades a la hora de comunicarse con lenguaje natural, por esta raz�n se utilizan los tableros de comunicaci�n basados en pictogramas. Estos tableros presentan siempre un mismo formato dependiendo del tipo de mensaje que se quiere transmitir, por ejemplo en agendas, secuencias de actividades , \dots Actualmente a pesar del avance de la tecnolog�a en muchos centros se siguen creando estos tableros de manera manual, ya que no hay una herramienta que se adapte a las necesidades de los usuarios. Por ello surge la necesidad de desarrollar una aplicaci�n que permita a los usuarios generar digitalmente estos tableros de comunicaci�n y que esta aplicaci�n sea accesible desde cualquier dispositivo.

El objetivo principal de este proyecto era desarrollar una herramienta que permitiese a los usuarios digitalizar los tableros que utilizan en su d�a a d�a, para ello hay que generar una plantilla que se podr� adaptar a cualquier tablero, cambiando unicamente los pictogramas oportunos. Para que el material generado se adaptase lo m�ximo posible a las necesidades de los usuarios se utiliz� una librer�a que permit�a arrastrar y soltar los elementos a cualquier posici�n y cambiar el tama�o de los mismos, as� pod�an crear cualquier tipo de material a partir de un lienzo en blanco.

Una vez desarrollada la parte que permit�a a los usuarios generar las plantillas, para facilitar la gesti�n del material generado por esta aplicaci�n se decidi� que era necesario que la herramienta se encargase de listar el material generado y permitiese realizar distintas acciones sobre el mismo.

%-------------------------------------------------------------------
\section{Trabajo futuro}
%-------------------------------------------------------------------
\label{cap6:sec:trabajofuturo}

Una vez terminado el proyecto, existen algunas carencias que se pueden suplir y as� dotar el proyecto de mayor funcionalidad y que as� el proyecto tenga un mayor alcance.

\begin{itemize}
	\item Implementar un buscador por nombre en los listados, as� cuando los usuarios tengan mucho material pueden buscar por nombre f�cilmente.
	\item Definir categor�as, para filtrar el contenido de las listas. Esto seria muy �til en el caso de que el contenido publico sea demasiado y se quiera buscar un tipo de contenido concreto, por ejemplo: normas, agendas, calendarios, etc.
	\item Vincular unas plantillas con otras, es decir, que al seleccionar un pictograma asociado a otra plantilla te lleve a ella.
	\item Permitir a los usuarios utilizar im�genes propias, en las plantillas y tableros.
	\item Que el usuario pueda interactuar con el tablero marcando el pictograma elegido.
	\item Mejorar el buscador de tal manera que al buscar la palabra ``hola'' los resultados de la b�squeda sean todos aquellos que empiezan por la palabra buscada, en este caso ``hola'' y ``holanda''.
\end{itemize}

% Variable local para emacs, para  que encuentre el fichero maestro de
% compilaci�n y funcionen mejor algunas teclas r�pidas de AucTeX
%%%
%%% Local Variables:
%%% mode: latex
%%% TeX-master: "../Tesis.tex"
%%% End:

