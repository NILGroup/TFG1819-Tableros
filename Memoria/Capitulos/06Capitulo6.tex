%---------------------------------------------------------------------
%
%                          Cap�tulo 6
%
%---------------------------------------------------------------------

\setlength{\parskip}{10pt}
\chapter{Conclusiones y trabajo futuro}

En este capitulo se muestran las conclusiones extra�das del TFG y el trabajo futuro que podr�a implementarse extendiendo la funcionalidad de este proyecto.

%-------------------------------------------------------------------
\section{Conclusiones}
%-------------------------------------------------------------------
\label{cap6:sec:conclusiones}



%-------------------------------------------------------------------
\section{Trabajo futuro}
%-------------------------------------------------------------------
\label{cap6:sec:trabajofuturo}

Una vez terminado el proyecto, existen algunas carencias que se pueden suplir y as� dotar el proyecto de mayor funcionalidad y que as� el proyecto tenga un mayor alcance.

\begin{itemize}
	\item Implementar un buscador por nombre en los listados, as� cuando los usuarios tengan mucho material pueden buscar por nombre f�cilmente.
	\item Definir categor�as, para filtrar el contenido de las listas. Esto seria muy �til en el caso de que el contenido publico sea demasiado y se quiera buscar un tipo de contenido concreto, por ejemplo: normas, agendas, calendarios, etc.
	\item Vincular unas plantillas con otras, es decir, que al seleccionar un pictograma asociado a otra plantilla te lleve a ella.
	\item Permitir a los usuarios utilizar im�genes propias, en las plantillas y tableros.
	\item Que el usuario pueda interactuar con el tablero marcando el pictograma elegido.
	\item Mejorar el buscador de tal manera que al buscar la palabra ``hola'' los resultados de la b�squeda sean todos aquellos que empiezan por la palabra buscada, en este caso ``hola'' y ``holanda''.
\end{itemize}

% Variable local para emacs, para  que encuentre el fichero maestro de
% compilaci�n y funcionen mejor algunas teclas r�pidas de AucTeX
%%%
%%% Local Variables:
%%% mode: latex
%%% TeX-master: "../Tesis.tex"
%%% End:

