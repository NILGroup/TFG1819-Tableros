%---------------------------------------------------------------------
%
%                          Capitulo 7
%
%---------------------------------------------------------------------
\setlength{\parskip}{10pt}
\chapter{Conclusions and Future work}

\begin{resumen}
	In this chapter in Section \ref{cap7:sec:conclusions} the conclusions drawn from the work presented in this report and in Section \ref{cap7:sec:futurework} future work are shown.
\end{resumen}

%-------------------------------------------------------------------
\section{Conclusions}
%-------------------------------------------------------------------
\label{cap7:sec:conclusions}

The main goal of this project was to develop a web application that would allow users to digitally create the boards they use in their day to day. To do this, it was necessary to create the possibility of creating templates that would serve as the basis for creating boards. In order that the application created was adapted as much as possible to the needs of the users, a library was used that allowed to drag and drop the elements to any position and change the size of the same, thus allowing the creation of any type of material from a blank canvas. The application was created following a user-centered design. The design of the application was created with the end users who also participated in the evaluation of the application. This evaluation has made it possible to verify the usefulness of the application, as well as to identify the aspects that need improvement.

The second objective of this TFG was to apply some of the knowledge acquired during the degree, especially useful have been the following subjects:

\begin{itemize}
	\item \textbf{Web Applications}: where I acquired the basic knowledge of HTML, CSS and JavaScript, necessary to create the application.
	\item \textbf{Interactive Systems Design}: where I learned to design user-centered applications, testing with interactive prototypes, evaluations on the final application, etc.
\end {itemize}

Finally, the third objective was to learn new things and this project has allowed me to deepen the use of JavaScript by fully exploring its functionalities and working with different libraries such as \textit{interact.js} and \textit{html2canvas}. It has also allowed me to learn to work with new technologies such as Firebase and LocalStorage.

%-------------------------------------------------------------------
\section{Future Work}
%-------------------------------------------------------------------
\label{cap7:sec:futurework}

Once the project is finished, there are several improvements that can be added in the future and thus provide the application with greater functionality:

\begin{itemize}
	\item Implement a search engine by the title of the template or board in the listings. So when users have a lot of material they can search by name easily.
	\item Define categories to filter the content of template and panel lists. This would be very useful if the public content is too large and you want to find a specific type of content, for example: rules, agendas, calendars, etc.
	\item Link some templates with others. For example, in the Point System template, users could link a board to each point and in the associated board show the activity they have to perform to get a reward.
	\item Allow users to use their own images, in the templates and boards. So users can further customize templates and boards, for example for users it is better to put a picture of their mother, than the mother pictogram.
	\item Allow the user to interact with the board by marking the chosen pictogram. For example, if the board of choosing toys is shown, the toy chosen by the user is marked on the board.
	\item Improve the search engine in such a way that when searching for the word ``hello'', for example, the search results are all those that start with the searched word, in this case ``hello'' and ``holland'' and not just the pictograms with the exact match.
	\item Add a button in the list of contents, which allows users to easily print their templates and boards.
	\item Adjust the text within the elements, so when an element is small the text should be adjusted to its size.
\end{itemize}


% Variable local para emacs, para  que encuentre el fichero maestro de
% compilaci�n y funcionen mejor algunas teclas r�pidas de AucTeX
%%%
%%% Local Variables:
%%% mode: latex
%%% TeX-master: "../Tesis.tex"
%%% End: