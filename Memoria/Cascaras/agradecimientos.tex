%---------------------------------------------------------------------
%
%                      agradecimientos.tex
%
%---------------------------------------------------------------------
%
% agradecimientos.tex
% Copyright 2009 Marco Antonio Gomez-Martin, Pedro Pablo Gomez-Martin
%
% This file belongs to the TeXiS manual, a LaTeX template for writting
% Thesis and other documents. The complete last TeXiS package can
% be obtained from http://gaia.fdi.ucm.es/projects/texis/
%
% Although the TeXiS template itself is distributed under the 
% conditions of the LaTeX Project Public License
% (http://www.latex-project.org/lppl.txt), the manual content
% uses the CC-BY-SA license that stays that you are free:
%
%    - to share & to copy, distribute and transmit the work
%    - to remix and to adapt the work
%
% under the following conditions:
%
%    - Attribution: you must attribute the work in the manner
%      specified by the author or licensor (but not in any way that
%      suggests that they endorse you or your use of the work).
%    - Share Alike: if you alter, transform, or build upon this
%      work, you may distribute the resulting work only under the
%      same, similar or a compatible license.
%
% The complete license is available in
% http://creativecommons.org/licenses/by-sa/3.0/legalcode
%
%---------------------------------------------------------------------
%
% Contiene la p�gina de agradecimientos.
%
% Se crea como un cap�tulo sin numeraci�n.
%
%---------------------------------------------------------------------

\chapter*{Agradecimientos}

\cabeceraEspecial{Agradecimientos}

En primer lugar agradecer a todas y cada una de las personas que he tenido la suerte de ser su alumna, cada uno ha dejado huella aportando su granito de arena en este camino, ayud�ndome a crecer profesional y personalmente, en especial a mis directoras, Raquel y Virginia, que han pasado a formar parte de mi vida durante este �ltimo a�o, dedic�ndome su tiempo y esfuerzo en este proyecto, que no habr�a sido posible sin sus directrices.

Tambi�n quiero hacer una menci�n especial a la Asociaci�n Autismo Sevilla y al colegio Angel Riviere, por ayudarnos a entender un poquito mas las necesidades de los usuarios d�ndonos consejos e ideas en el desarrollo del trabajo, colaborando adem�s en la realizaci�n de las evaluaciones de la aplicaci�n. Gracias de todo coraz�n.

Para acabar quiero agradecer a mi familia y amigos la paciencia que han tenido y los �nimos que me han brindado en este ultimo a�o, haci�ndome sacar fuerzas en los momentos mas dif�ciles para seguir adelante.

%Agradecer a directoras de proyecto

%agradecer a la asociacion y en especial al colegio



\endinput
% Variable local para emacs, para  que encuentre el fichero maestro de
% compilaci�n y funcionen mejor algunas teclas r�pidas de AucTeX
%%%
%%% Local Variables:
%%% mode: latex
%%% TeX-master: "../Tesis.tex"
%%% End:
