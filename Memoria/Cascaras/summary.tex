%---------------------------------------------------------------------
%
%                      resumen.tex
%
%---------------------------------------------------------------------
%
% Contiene el cap�tulo del resumen.
%
% Se crea como un cap�tulo sin numeraci�n.
%
%---------------------------------------------------------------------

\chapter*{Abstract}
\cabeceraEspecial{Abstract}

People with cognitive disabilities usually present various difficulties in the use of natural language, from problems of comprehension of oral or written language to difficulties to express themselves with their environment and need to seek support to help them obtain effective communication. One of the most used supports are the pictograms (graphic symbols used to represent objects, actions, \ldots). In communication with pictograms it is usual to use so-called communication boards (surface on which the pictograms are arranged to communicate). Communication boards allow you to represent the message you want to transmit using pictograms. Technologies have advanced a lot in this field, and there are many tools for the creation of communication boards. But despite the variety of existing tools, end users often have difficulty finding one that suits their needs completely, since they are quite limited when generating boards and often choose to generate these boards manually with pens and paper. A very common problem in existing solutions is that you have to create boards from scratch or using very general and equal templates for everyone (agenda, calendar, rules, \ldots).

The objective of this work has been to develop a web application that allows parents, tutors and teachers of people with cognitive disabilities to generate personal templates that can then be reused in the creation of boards and thus their creation. It has been designed as a web application to be accessible from any device and thus reach the largest number of users possible. The application is developed in such a way that the users can generate the boards without any type of restriction, allowing to change the position and the size of the elements that form them.

Once the application has been developed, it has been subjected to an evaluation with final users that has allowed us to know the usefulness and usability of the application. The result of this evaluation shows that the application is useful and can be adapted to the needs of the users, although it can be improved in some aspects.

\section*{Keywords}

\noindent Communication Board, Pictograms, Cognitive Disability, Web Application, Accessibility, Templates.

\endinput
% Variable local para emacs, para  que encuentre el fichero maestro de
% compilaci�n y funcionen mejor algunas teclas r�pidas de AucTeX
%%%
%%% Local Variables:
%%% mode: latex
%%% TeX-master: "../Tesis.tex"
%%% End:

