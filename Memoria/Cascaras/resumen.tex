%---------------------------------------------------------------------
%
%                      resumen.tex
%
%---------------------------------------------------------------------
%
% Contiene el cap�tulo del resumen.
%
% Se crea como un cap�tulo sin numeraci�n.
%
%---------------------------------------------------------------------

\chapter*{Resumen}
\cabeceraEspecial{Resumen}

Las personas con discapacidad cognitiva suelen presentar diversas dificultades en el uso del lenguaje natural, desde problemas de comprensi�n del lenguaje oral o escrito hasta dificultades para expresarse con su entorno y necesitan buscar apoyos que les ayuden a obtener una comunicaci�n efectiva. Uno de los apoyos mas utilizado son los pictogramas (s�mbolos gr�ficos que sirven para representar objetos, acciones,\ldots). En la comunicaci�n con pictogramas es habitual el uso de los llamados tableros de comunicaci�n (superficie sobre la que se disponen los pictogramas para comunicarse). Aunque existen multitud de herramientas destinadas a la creaci�n de tableros de comunicaci�n, los usuarios finales suelen tener dificultades para encontrar una que se adapte a sus necesidades encontr�ndose con bastantes limitaciones a la hora de generar los tableros. Un problema muy com�n en las soluciones existentes radica en que hay que crear siempre los tableros desde cero o usando plantillas muy generales e iguales para todos (agenda, calendario, normas, \ldots).

El objetivo de este trabajo ha sido desarrollar una herramienta que permite a padres, tutores y profesores de personas con discapacidad cognitiva generar plantillas personales que luego podr�n ser reutilizadas en la creaci�n de tableros y as� agilizar su creaci�n. Esta herramienta se ha dise�ado como una aplicaci�n web para que sea accesible desde cualquier dispositivo y as� llegar al mayor numero de usuarios posible. La aplicaci�n est� desarrollada de tal manera que los usuarios pueden generar los tableros sin ning�n tipo de restricci�n, permitiendo cambiar la posici�n y el tama�o de los elementos que forman los mismos.

Una vez desarrollada la aplicaci�n, ha sido sometida a una evaluaci�n con usuarios finales que nos ha permitido saber que la aplicaci�n es de utilidad y se puede adaptar a las necesidades de los usuarios, aunque es mejorable en algunos aspectos. 

\section*{Palabras clave}

\noindent Tablero de comunicaci�n, Pictogramas, Discapacidad cognitiva, Aplicaci�n Web, Accesibilidad, Plantillas.

\endinput
% Variable local para emacs, para  que encuentre el fichero maestro de
% compilaci�n y funcionen mejor algunas teclas r�pidas de AucTeX
%%%
%%% Local Variables:
%%% mode: latex
%%% TeX-master: "../Tesis.tex"
%%% End:
