%---------------------------------------------------------------------
%
%                      resumen.tex
%
%---------------------------------------------------------------------
%
% Contiene el cap�tulo del resumen.
%
% Se crea como un cap�tulo sin numeraci�n.
%
%---------------------------------------------------------------------

\chapter*{Resumen}
\cabeceraEspecial{Resumen}

Las personas con discapacidad cognitiva suelen presentar dificultades en el uso del lenguaje natural, para superar estas barreras hacen uso de tableros de comunicaci�n. Los tableros de comunicaci�n permiten representar el mensaje que se quiere transmitir mediante im�genes y pictogramas. Las tecnolog�as han avanzado mucho en este campo, y existen multitud de herramientas destinadas a la creaci�n de tableros, pero a pesar de la variedad de herramientas, los usuarios no han encontrado una que se adapte a sus necesidades, ya que se encuentran bastantes limitaciones a la hora de generar un tablero y finalmente optan por generar estos tableros manualmente con bol�grafos y papel.

El objetivo de este trabajo es desarrollar una herramienta que permita a padres, tutores y profesores de personas con discapacidad cognitiva generar tableros y plantillas y reutilizarlos f�cilmente, as� agilizar la creaci�n de tableros a partir de una plantilla. Se ha dise�ado como una aplicaci�n web para que sea accesible desde cualquier dispositivo y as� llegue al mayor numero de usuarios posible. La aplicaci�n esta desarrollada de tal manera que los usuarios pueden generar los tableros sin ning�n tipo de restricci�n, permitiendo cambiar la posici�n y el tama�o de los elementos que forman los tableros.

Una vez desarrollada la aplicaci�n, fue sometida a una evaluaci�n con los usuarios finales que nos permit�a conocer la utilidad y usabilidad de la aplicaci�n. El resultado de esta evaluaci�n refleja que la aplicaci�n es de utilidad y se adapta a las necesidades de los usuarios. 

\section*{Palabras clave}

Tablero de comunicaci�n, Pictogramas, Discapacidad cognitiva, 

\endinput
% Variable local para emacs, para  que encuentre el fichero maestro de
% compilaci�n y funcionen mejor algunas teclas r�pidas de AucTeX
%%%
%%% Local Variables:
%%% mode: latex
%%% TeX-master: "../Tesis.tex"
%%% End:
